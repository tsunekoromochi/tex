\chapter{結論}
本研究の目的であるLab空間におけるグラデーションプレートを用いた色変換処理を行うことを可能とした。\par
実験結果より、Lab色空間において色変換を行うとき、従来法であるL*、a*、b*の値での標準偏差を用いて変換するよりグラデーションプレートを用いて変換する方が望ましいという事がわかった。従来法でも色変換をすることによって参照画像のイメージを転写することは可能ではあったが、Lab色空間において色相はa*=0、b*=0からの角度で決まることからL、a*、b*の値での標準偏差を用いることで変換後の値が理想の値とは大きく異なってしまうことから標準偏差を用いて色変換を行うのは良くないことがわかった。一方、グラデーションプレートを用いることでL、a*、b*の値での標準偏差に関係なく、誤差を抑えたまま色変換が行えることがわかった。一方、Lab色空間上のb軸上に色分布がある場合はグラデーションプレートを用いると色変換がうまくいかないということがある。\par
今後の課題としてLab色空間において色変換をする際、特徴領域の色分布においてb軸上に色分布がある場合でも色変換がうまく行えるようにしたい。そのためには他の変換法や場合分けなどが必要になると考える。その問題を解決することにより画像のイメージとしてより多くのイメージを転写できることができ、多くのの転写した画像ができることとなる。そのことにより、より多くのイメージを持ったデザインができ、人の伝えたいイメージである意図やメッセージの選択が多くなることがわかる。結果、イメージの選択肢が増え、より正確メッセージを伝えることができるのではないかと考える。
