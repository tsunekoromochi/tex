\chapter{序論}


近年、デジタルコンテンツから情報を得たり、デジタルコンテンツを発信したりすることが非常に身近になった。その反面、世の中にあふれるデジタルコンテンツが膨大になり、デジタルコンテンツを制作・発信しても膨大なデジタルコンテンツの山に埋もれて情報を届けたい相手に届かない場合もある。\par
膨大なデジタルコンテンツに埋もれることなく、情報を伝えたい相手に的確に届けるための手段の1つとして、デジタルコンテンツのデザインに製作者の意図やメッセージを反映できるよう工夫することが挙げられる。\par
本研究では、画像の色彩という情報に着目して、Lab色空間で画像の色を変換させ、色変換において人間の感性を思い通りに移し替えることを狙いとする。それを達成するために本研究ではグラデーションプレートを用いた色変換法の実装を目指す。将来的には、デジタルコンテンツの製作者が、確実に受け手に伝わるデジタルコンテンツを作成するための手助けになることを期待する。


\section{はじめに}
\subsection{背景}
近年では、インターネットやPC、スマートデバイスなどの普及により、デジタルコンテンツに手軽にアクセスし、情報収集に利用する機会が増え、また、それらを誰もが気軽に作成し、発信することができるようなった。ただその反面、情報の受け手が目にする情報量も膨大になり、情報を発信しても製作者が伝えたい意図やメッセージを受け手に届けることなく、膨大な情報の山に埋もれてしまうのも事実である。そのような状況の中で、製作者の伝えたい意図やメッセージをより的確に受け手に伝えるための手段の1つとして、デジタルコンテンツのデザインを考える際に、製作者の伝えたい意図やメッセージを反映したデザインを選択もしくは作成することが挙げられる。\par
人間が外界から受ける刺激によって生じる感覚は視覚、聴覚、味覚、嗅覚、触覚の5つが存在する。それらの感覚の中でも視覚は、人間の得る情報の8割を司るため、最も重要な要素であるといえる。特に多くのデジタルコンテンツでは嗅覚、味覚が存在しないため、視覚による情報はより大きな構成要素になると考えられる。そのため、作成したデジタルコンテンツによって製作者の伝えたい意図や、メッセージをより効果的に伝えるには視覚情報のデザインに工夫をする必要があると考える。 \par
視覚情報に関するデザインの構成要素には色彩、レイアウト、形状、テクスチャ、文字のフォントなどがあるが、本研究では上記の構成要素の中の色彩という要素に着目して、色変換処理を施し、色変換処理における人間の感性に与える影響を考察することを行う。具体的には、従来法\cite{1}であるLab色空間においてL*、a*、b*の値での標準偏差を用いた変換法と本研究の提案手法であるグラデーションプレートを用いた変換法の2つの変換処理による人間の感性に与える影響の比較を行う。

\newpage
\subsection{本研究の狙い}
従来法であるL*、a*、b*の値での標準偏差を用いた変換法に加えて、提案手法として、新たにグラデーションプレートを用いた変換法を用いた画像変換処理をできるようにすることが本研究の狙いである。これによって、L*、a*、b*の値での標準偏差を用いた変換法で変換した画像よりもグラデーションプレートを用いた変換法で画像のほうが、より製作者のデザインしたデジタルコンテンツが製作者の意図した印象やメッセージを的確に受け手に与えることができるものなのかどうかを判断することができる。すなわち、より人間の感性に響き、膨大な情報の山の中に埋もれず相手に届くデジタルコンテンツを作成するための手助けになると考える。

