\chapter{原理}

\section{Lab色空間}
Lab色空間とは補色空間の一種で、明度を意味する次元Lと補色次元のaおよびbを持ち、CIE XYZ 色空間の座標を非線形に圧縮したものに基づいている空間である。\par
XYZ 色空間とは電磁波の可視スペクトル域における物理的な色と、心理学的な人間の色覚における知覚色との間の関係を初めて定量的に定義した色空間であり、sRGBやadobeRGBの色空間を他の色空間に変換するときの中間作業空間として広く使われる。\par
3刺激を値をそのまま用いたXYZ表色系では人が識別できる色差が色の領域によってかなり異なることから、できるだけ人間の感覚に近い均等な色空間として考案されたのがLab表色系である。\cite{2}
\newpage
Lab色空間ではa* の+方向が赤、-方向が緑、b* の+方向が黄、-方向が青、という配置になっており、人間が認識する心理的な補色の関係(赤と緑、青と黄)になっている。真ん中の a*=0,  b*=0が無彩色である。 a* や b* の絶対値が大きく,中心から遠い色ほど彩度が高い。また,a* と b* の値で決まる方向,色相に当たる。\cite{3}\par
Lab色空間を地球儀になぞらえると、地軸が明度軸、経度方向が色相の配置になっている。また、地球儀を地軸に垂直な平面で輪切りした断面がa* b*色度図となる。\par
\begin{figure}[htbp]
  \begin{center}
    \includegraphics[clip,width=7.0cm]{./chapter2/lab.eps}
    \caption{Lab色空間}
    \label{fig:lab}
  \end{center}
\end{figure}
L*, a*, b* の値は,XYZ の値を,視覚の特性を表現した簡単な変換式で変換することによって計算される。具体的には,Y を変換したものを  L* (明度) とし, 差 [(Xの変換)-(Yの変換)] を定数倍したものを a*,差 [(Yの変換)-(Zの変換)] を定数倍したものを b* としている。L* は Lightness の略で,明度にあたる。0~100 の数値をとる。\par

\newpage
\subsection{Lab空間の利点}
RGBやCMYKとは異なり、Lab色空間は人間の視覚を近似するよう設計されている。知覚的均等性を重視しており、L成分値は人間の明度の知覚と極めて近い。したがって、カラーバランス調整を正確に行うために出力曲線をaおよびbの成分で表現したり、コントラストの調整のためにL成分を使ったりといった利用が可能である。また、Lab色空間はコンピュータディスプレイやプリンタや人間の知覚よりも色域が広く、Lab色空間で表現したビットマップ画像は同等精度のRGBやCMYKのビットマップ画像よりもピクセル当たりのデータ量が多くなる。\par
また、図\ref{fig:lab2}[1]はxy色空間における色差の識別限界を楕円で表したものであり、図\ref{fig:lab2}[2]はLab色空間における色差の識別限界を楕円で表したものである。\par
\begin{figure}[htbp]
  \begin{center}
    \begin{tabular}{c}

      % 1
      \begin{minipage}{0.33\hsize}
        \begin{center}
          \includegraphics[clip, width=5.0cm]{./lab.eps}
          \hspace{1.6cm} [1]xy色空間
        \end{center}
      \end{minipage}

      % 2
      \begin{minipage}{0.33\hsize}
        \begin{center}
          \includegraphics[clip, width=5.0cm]{./lab2.eps}
          \hspace{1.6cm} [2]Lab色空間
        \end{center}
      \end{minipage}

    \end{tabular}
    \caption{人が識別できる色差の識別限界を表す楕円}
    \label{fig:lab2}
  \end{center}
\end{figure}
図\ref{fig:lab2}[1]のXY色空間では場所によって楕円のサイズがかなり異なっているが、図\ref{fig:lab2}[2]のLab色空間では場所による楕円のサイズの違いが小さくて色空間が均等になっていることからLab色空間では様々な色に対応できる。
\newpage
\section{色変換}
本研究で扱う対象は、画像の入出力ともRGBデータからなるデジタルコンテンツを想定している。所望の色調を持つ画像を参照画像として用意し、参照画像の色を転写する画像を対象画像として用意する。参照画像のグラデーション分布に基づく色彩情報を対象画像の色変換に利用し、色変換前後の感性に与える影響の比較を行う。

\newpage
\section{L*、a*、b*の値での標準偏差を用いた色変換}
\subsection{分散}
分散とはデータの散らばり具合を表す指標であり分散の値が大きいと平均から遠く離れたデータが多く、散らばりが大きいことを示す。一方、分散の値が小さいと平均から近いデータが多く、散らばりが小さいことを示す。散らばりが小さいと分散は0に近づく。分散を求めるには偏差(それぞれの数値と平均との差)を2乗し、平均を取ることにより求められる。\cite{4}\par
式で表すと以下のようになる。\par
\begin{eqnarray}
  s^2 = \frac{1}{n}\sum_{n=1}^n (x_i-\overline{x})^2
\end{eqnarray}
\[
nはデータの個数,x_iは個々の数値,\overline{x}は平均値
\]
\subsection{標準偏差}
標準偏差とはデータの散らばり具合を表す指標であり分散と同様に標準偏差の値が大きいと平均から遠く離れたデータが多く、散らばりが大きいことを示し、標準偏差の値が小さいと平均から近いデータが多く、散らばりが小さいことを示す。標準偏差を求めるには分散の正の平方根を取ることで求められる。\cite{5}\par
式で表すと以下のようになる。\par
\begin{eqnarray}
 s &=& \sqrt{s^2}\\
  &=& \sqrt{\frac{1}{n}\sum_{n=1}^n (x_i-\overline{x})^2}
\end{eqnarray}
\[
s^2は分散、nはデータの個数、x_iは個々の数値、\overline{x}は平均値
\]

\newpage
\subsection{変換法}
 Lab色空間は0~100の値からなる明るさを示すL軸、緑〜赤の色の要素を示すa軸、青〜黄の色の要素を示すb軸の3つの要素で色の数値や領域を表すカラーモデルである。Lab色空間は人間の視覚に近似するように設計されており、人間の目が捉えられる色全体を定義している。そのため、Lab色空間は表現できる色の範囲が広いのが特徴である。以下にLab色空間に基づく色変換の手法を示す。\par
対象画像の各画素における明るさの要素をl、a軸の要素をa、b軸の要素をbとし、それらの平均を<l>、<a>、<b>として各画素のそれぞれの要素から平均を減算する。

\begin{eqnarray}
l^{\verb|'|} &=& l-<l> \\
a^{\verb|'|} &=& a-<a> \\
b^{\verb|'|} &=& b-<b> 
\end{eqnarray}

次に各要素の減算結果に対象画像の各要素の標準偏差で参照画像の各要素の標準偏差を割ったものをかける。
\begin{eqnarray}
  l^{\verb|''|} &=& \frac{\sigma^l_s}{\sigma^l_t}l^{\verb|'|}\\
  a^{\verb|''|} &=& \frac{\sigma^a_s}{\sigma^a_t}a^{\verb|'|}\\
  b^{\verb|''|} &=& \frac{\sigma^b_s}{\sigma^b_t}b^{\verb|'|}
\end{eqnarray}
\[
(\sigma^l_tはlの対象画像の標準偏差、\sigma^l_sはlの参照画像の標準偏差)
\]
\newpage
最後に各要素の演算結果と参照画像の各要素の平均である$<l^{\verb|'|}>、<a^{\verb|'|}>、<b^{\verb|'|}>$の和を求める。

\begin{eqnarray}
l^{\verb|'''|} &=&  l^{\verb|''|} + <l^{\verb|'|}> \\
a^{\verb|'''|} &=&  a^{\verb|''|} + <a^{\verb|'|}> \\
b^{\verb|'''|} &=&  b^{\verb|''|} + <b^{\verb|'|}> 
\end{eqnarray}

$l^{\verb|'''|}、a^{\verb|'''|}、b^{\verb|'''|}$はそれぞれ色変換後の対象画像の各画素における明るさ、a軸、b軸の要素を意味している。

\subsection{問題点}
標準偏差とはデータの散らばり具合を表す指標であり、遠く離れたデータが多いと標準偏差の値は大きくなり、近いデータが多いと標準偏差の値は小さくなる。色変換を行うにおいて上式の場合だと標準偏差の値は色変換において大きく左右することがわかる。色変換を行うにあたって標準偏差の値が大きいと色を変換するときに参照画像の色情報と大きく異なる結果となってしまう問題がある。よって、標準偏差を用いて色変換を行う際には標準偏差の値が小さい、できるだけ0に近いものが望ましい。\par
